% Copyright 2012 David W. Hogg (NYU).  All rights reserved.

\documentclass[pdftex,aspectratio=169]{beamer}
\usepackage{amssymb,amsmath,mathrsfs}
\usecolortheme{default}
\usefonttheme{serif}
\setbeamertemplate{navigation symbols}{} % remove navigation symbols
\setbeamercolor{footline}{fg=gray, bg=gray}

% this one is debatable
\renewcommand{\emph}[1]{\textbf{#1}}

\setbeamertemplate{footline}[text line]{%
  \parbox{\linewidth}{\sffamily\insertshorttitle\hfill\insertshortauthor\hfill\insertdate\hfill\insertpagenumber}}
\setbeamertemplate{navigation symbols}{}

%%% color commands
\newcommand{\whiteonblack}{%
  \colorlet{fg}{white}
  \colorlet{bg}{black}
  \setbeamercolor{normal_text}{fg=white,bg=black}
  \setbeamercolor{background canvas}{fg=white,bg=black}
  \setbeamercolor{alerted_text}{fg=yellow}
  \setbeamercolor{example_text}{fg=white}
  \setbeamercolor{structure}{fg=white}
  \setbeamercolor{palette_quaternary}{fg=white}
}
\newcommand{\blackonwhite}{%
  \colorlet{fg}{black}
  \colorlet{bg}{white}
  \setbeamercolor{normal_text}{fg=black,bg=white}
  \setbeamercolor{background canvas}{fg=black,bg=white}
  \setbeamercolor{alerted_text}{fg=blue}
  \setbeamercolor{example_text}{fg=black}
  \setbeamercolor{structure}{fg=black}
  \setbeamercolor{palette_quaternary}{fg=black}
}
\xdefinecolor{pink}{rgb}{1.0,0.9,0.9}

%%% size and shape commands
\newlength{\figurewidth}
\setlength{\figurewidth}{0.9\textwidth}
\newlength{\figureheight}
\setlength{\figureheight}{0.8\textheight}

%%% text commands
\frenchspacing
\newcommand{\project}[1]{\textsl{#1}}
  \newcommand{\apogee}{\project{APOGEE}}
  \newcommand{\an}{\project{Astrometry.net}}
  \newcommand{\tc}{\project{The~Cannon}}
  \newcommand{\euclid}{\project{Euclid}}
  \newcommand{\flickr}{\project{flickr}}
  \newcommand{\gaia}{\project{Gaia}}
  \newcommand{\galex}{\project{GALEX}}
  \newcommand{\kepler}{\project{Kepler}}
  \newcommand{\GALEX}{\galex}
  \newcommand{\hst}{\project{HST}}
  \newcommand{\hipparcos}{\project{Hipparcos}}
  \newcommand{\lsst}{\project{LSST}}
  \newcommand{\sdss}{\project{SDSS}}
  \newcommand{\sdssiii}{\project{SDSS-III}}
  \newcommand{\sdssiv}{\project{SDSS-IV}}
  \newcommand{\boss}{\sdssiii\ \project{BOSS}}
  \newcommand{\osss}{\project{OSSS}}
  \newcommand{\ska}{\project{SKA}}
  \newcommand{\vo}{\project{VO}}
  \newcommand{\rttd}{\project{Right Thing To Do}$^{\mbox{\scriptsize\sffamily{TM}}}$}
\newcommand{\foreign}[1]{\textit{#1}}
\newcommand{\latin}[1]{\foreign{#1}}
  \newcommand{\cf}{\latin{cf.}}
  \newcommand{\eg}{\latin{e.g.}}
  \newcommand{\etal}{\latin{et~al.}}
  \newcommand{\etc}{\latin{etc.}}
  \newcommand{\ie}{\latin{i.e.}}
  \newcommand{\vs}{\latin{vs.}}

%%% math-mode commands
\newcommand{\mmatrix}[1]{\boldsymbol{#1}}
\newcommand{\tv}[1]{\boldsymbol{#1}}
\newcommand{\dd}{\mathrm{d}}
\newcommand{\given}{\,|\,}
\newcommand{\transpose}[1]{{#1}^{\mathsf{T}}}
\newcommand{\inverse}[1]{{#1}^{-1}}

%%% units
\newcommand{\unit}[1]{\mathrm{#1}}
\newcommand{\rad}{\unit{rad}}
\newcommand{\s}{\unit{s}}
\newcommand{\yr}{\unit{yr}}
\newcommand{\km}{\unit{km}}
\newcommand{\kmps}{\km\,\s^{-1}}

%%% symbols
\newcommand{\Teff}{T_{\mathrm{eff}}}
\newcommand{\teff}{\Teff}
\newcommand{\logg}{\log g}
\newcommand{\feh}{[\mathrm{Fe}/\mathrm{H}]}
\newcommand{\vsini}{v\,\sin i}
 % hogg standard colors

\title[Causal Structure]{Data-driven methods and causal structure}
\author[David W. Hogg \textsl{(NYU) (Flatiron) (MPIA)}]{David W. Hogg \\
  \textsl{\footnotesize Center for Cosmology and Particle Physics, Department of Physics, New York University}\\
  \textsl{\footnotesize Flatiron Institute }\\
  \textsl{\footnotesize Max-Planck-Institut f\"ur Astronomie}}
\date{2019 December 13}

\newcommand{\conclusion}{
\begin{frame}
  \frametitle{punchlines}
  \begin{itemize}
  \item Huge data sets create new opportunities.
    \begin{itemize}
    \item they also present new challenges
    \end{itemize}
  \item Industrial (dot-com) methods are brilliant.
    \begin{itemize}
    \item they can only solve a very limited set of problems
    \end{itemize}
  \item We won't reap the full benefit of larger data sets without new technology.
    \begin{itemize}
    \item call to arms
    \item (and get rich too!)
    \end{itemize}
  \end{itemize}
\end{frame}
}

\begin{document}

\begin{frame}[plain]
\maketitle
\end{frame}

\conclusion

\begin{frame}
  \frametitle{map--reduce or die}
  \begin{itemize}
  \item \textsl{``We won't even consider algorithms that can't be
    written in the map--reduce framework.''}
  \end{itemize}
\end{frame}

\conclusion

\end{document}

\begin{frame}
  \frametitle{map--reduce}
  \begin{itemize}
  \item map:
    \begin{itemize}
    \item at each ``data point'' (on a distributed system), do an
      operation on that datum, produce output
    \item \texttt{if "kittens" is in document: \\ ~~~~return (URL, PageRank)}
    \item \emph{distributed data} is the key: Every datum is near a
      CPU.
    \end{itemize}
  \item reduce:
    \begin{itemize}
    \item between each pair of outputs, do an operation and return one
      new output, recurse up the tree
    \item \texttt{if PageRank[0] > PageRank[1]: \\ ~~~~return (URL[0], PageRank[0]) \\ else: \\ ~~~~return (URL[1], PageRank[1])}
    \item \emph{tree structure} of the data center is the key: There are only
      $\log_2 N$ branches to any datum.
    \end{itemize}
  \end{itemize}
\end{frame}

\begin{frame}
  \frametitle{map--reduce}
  \includegraphics[scale=0.75]{../../pgm/mapreduce.pdf}
\end{frame}

\begin{frame}
  \frametitle{map--reduce}
  \begin{itemize}
  \item Brilliant.  And a huge opportunity.
    \begin{itemize}
    \item the \emph{load on your air conditioning} scales as $N\,\log N$
    \item but the \emph{time it takes} scales as just $\log N$
    \end{itemize}
  \item Map--reduce made internet search \emph{possible}.
    \begin{itemize}
    \item (goes by many names, some trade-marked!)
    \end{itemize}
  \end{itemize}
\end{frame}

\begin{frame}
  \frametitle{maximum-likelihood and map--reduce}
  \begin{itemize}
  \item full-data likelihood: $\displaystyle p(D\given\theta) =
    \prod_n p(d_n\given\theta)$
  \item Find a \emph{local maximum with respect to $\theta$} of this
    likelihood.
  \item map:
    \begin{itemize}
    \item compute $\displaystyle\frac{\dd \ln p(d_n\given\theta)}{\dd\theta}$
    \end{itemize}
  \item reduce:
    \begin{itemize}
    \item pairwise sum
    \end{itemize}
  \item Go uphill.  Repeat as necessary; each iteration only takes
    $\log N$ time.
    \begin{itemize}
    \item (use L-BFGS or conjugate-gradient or whatever you like)
    \item complexity is $N\,\log N$, \emph{compute time} is $\log N$
    \end{itemize}
  \item Local optimization typically takes $M^2\log N$ time, where $M$ is the \emph{number of parameters}.
    \begin{itemize}
    \item (maybe $M^3$ if you want a full error analysis)
    \end{itemize}
  \end{itemize}
\end{frame}

\begin{frame}
  \frametitle{all that matters is the number of parameters}
  \begin{itemize}
  \item So we are done, right?
  \item<2-> \emph{Nope.}
  \item<2-> In all \emph{real} problems, the number of parameters $M$ \emph{scales with the data volume $N$}.
  \item<2-> If you want to go beyond maximum-likelihood, things get \emph{hard}.
  \end{itemize}
\end{frame}

\begin{frame}
  \frametitle{physics problems are hierarchical}
  \includegraphics{../../pgm/weaklensing.pdf}
\end{frame}

\begin{frame}
  \frametitle{Bayesian inference \emph{isn't} map--reduce}
  \begin{itemize}
  \item $\displaystyle p(\theta\given D) = \frac{1}{Z}\,p(\theta)\,\prod_n p(d_n\given\theta)$
  \item map:
    \begin{itemize}
    \item compute functions $p(d_n\given\theta)$
    \end{itemize}
  \item reduce:
    \begin{itemize}
    \item product functions together (ending with the prior)
    \end{itemize}
  \item but think about how you pass forward those functions
    \begin{itemize}
    \item $\theta$ has $10^6$ or more parameters
    \item functions are multi-modal
    \item support is broader than Gaussian
    \item when the data get large, the \emph{resolution required} becomes unsustainable
    \end{itemize}
  \end{itemize}
\end{frame}

\begin{frame}
  \frametitle{even the frequentists are doomed}
  \begin{itemize}
  \item All the ``$M$ scaling with $N$'' arguments apply to frequentists and Bayesians alike.
  \item Computing the full-data likelihood \emph{function} is just as hard as computing the full-data posterior PDF.
    \begin{itemize}
    \item (local optimization of the likelihood is easy, full description of the function is hard)
    \end{itemize}
  \end{itemize}
\end{frame}

\begin{frame}
  \frametitle{marginalization is hard---and unavoidable}
  \includegraphics{../../pgm/weaklensing.pdf}
\end{frame}

\begin{frame}
  \frametitle{my approach}
  \begin{itemize}
  \item<1-> Brute Force (tm).
  \item<2-> Plus some help from applied math and computer vision.
    \begin{itemize}
    \item approximate Bayesian inference
    \item very clever Markov-Chain Monte Carlo methods
    \item exploiting problem sparsity
    \end{itemize}
  \end{itemize}
\end{frame}

\begin{frame}
  \frametitle{my day job}
  \begin{itemize}
  \item Lang \& Hogg {\small(forthcoming)}: a $10^9$-parameter model of the
    $10^{13}$ \project{Sloan Digital Sky Survey} pixels (\project{The Tractor})
  \item Brewer \etal\ {\small(forthcoming)}: Bayesian non-parametrics but with
    priors that represent our actual prior knowledge
  \item Foreman-Mackey \etal\ {\small(arXiv:1202.3665)}:
    \project{emcee}, the MCMC Hammer: flexible, parallelized, adaptive
    sampler
  \item Bovy, Murray, Hogg {\small(arXiv:0903.5308)}: a dynamical inference
    fully marginalizing out a non-parametric distribution function
  \item Bovy \etal\ {\small(arXiv:1105.3975)}: a 60,000-parameter model of
    700,000 flux measurements, followed by predictions for 160,000,000
    point sources
  \end{itemize}
\end{frame}

\begin{frame}
  \frametitle{it's actually worse than you think}
  \begin{itemize}
  \item ``More data'' means ``more new discoveries''.
  \item The \emph{number of hypotheses to test} also grows with the data size $N$.
  \item There will be far more hypotheses than physicists.
    \begin{itemize}
    \item (this is already true, of course)
    \item citizen science?
    \item robot science?
    \end{itemize}
  \end{itemize}
\end{frame}

\conclusion

\begin{frame}
  \frametitle{questions}
  \begin{itemize}
  \item Are we concerned about \emph{lossy steps} in data analysis?
    \begin{itemize}
    \item catalog generation, two-point function estimation
    \item principled approaches require likelihood functions
    \item can we \emph{require} projects to marginalize or to saturate Cram\'er-Rao bounds?
    \end{itemize}
  \item Can we more \emph{directly compare theory} with data?
    \begin{itemize}
    \item comparisons through ``summary statistics'' are bad!
    \item simulations are slow; put them in the inference loop?
    \item propagation of \emph{theory} uncertainties (known unknowns)?
    \end{itemize}
  \item How do we ensure \emph{flexibility of systems}?
    \begin{itemize}
    \item fast evolution in hardware, methodological, and scientific contexts
    \item usability, repeatability, adaptability
    \end{itemize}
  \item Who is going to \emph{do all this work}, and when?
    \begin{itemize}
    \item possible operations framework, or economic model?
    \item training and promotion issues; \emph{ethical issues}
    \item fastness
    \end{itemize}
  \end{itemize}
\end{frame}

\end{document}
